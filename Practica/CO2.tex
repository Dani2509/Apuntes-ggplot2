% Options for packages loaded elsewhere
\PassOptionsToPackage{unicode}{hyperref}
\PassOptionsToPackage{hyphens}{url}
\PassOptionsToPackage{dvipsnames,svgnames,x11names}{xcolor}
%
\documentclass[
  letterpaper,
  DIV=11,
  numbers=noendperiod]{scrartcl}

\usepackage{amsmath,amssymb}
\usepackage{lmodern}
\usepackage{iftex}
\ifPDFTeX
  \usepackage[T1]{fontenc}
  \usepackage[utf8]{inputenc}
  \usepackage{textcomp} % provide euro and other symbols
\else % if luatex or xetex
  \usepackage{unicode-math}
  \defaultfontfeatures{Scale=MatchLowercase}
  \defaultfontfeatures[\rmfamily]{Ligatures=TeX,Scale=1}
\fi
% Use upquote if available, for straight quotes in verbatim environments
\IfFileExists{upquote.sty}{\usepackage{upquote}}{}
\IfFileExists{microtype.sty}{% use microtype if available
  \usepackage[]{microtype}
  \UseMicrotypeSet[protrusion]{basicmath} % disable protrusion for tt fonts
}{}
\makeatletter
\@ifundefined{KOMAClassName}{% if non-KOMA class
  \IfFileExists{parskip.sty}{%
    \usepackage{parskip}
  }{% else
    \setlength{\parindent}{0pt}
    \setlength{\parskip}{6pt plus 2pt minus 1pt}}
}{% if KOMA class
  \KOMAoptions{parskip=half}}
\makeatother
\usepackage{xcolor}
\setlength{\emergencystretch}{3em} % prevent overfull lines
\setcounter{secnumdepth}{-\maxdimen} % remove section numbering
% Make \paragraph and \subparagraph free-standing
\ifx\paragraph\undefined\else
  \let\oldparagraph\paragraph
  \renewcommand{\paragraph}[1]{\oldparagraph{#1}\mbox{}}
\fi
\ifx\subparagraph\undefined\else
  \let\oldsubparagraph\subparagraph
  \renewcommand{\subparagraph}[1]{\oldsubparagraph{#1}\mbox{}}
\fi

\usepackage{color}
\usepackage{fancyvrb}
\newcommand{\VerbBar}{|}
\newcommand{\VERB}{\Verb[commandchars=\\\{\}]}
\DefineVerbatimEnvironment{Highlighting}{Verbatim}{commandchars=\\\{\}}
% Add ',fontsize=\small' for more characters per line
\usepackage{framed}
\definecolor{shadecolor}{RGB}{241,243,245}
\newenvironment{Shaded}{\begin{snugshade}}{\end{snugshade}}
\newcommand{\AlertTok}[1]{\textcolor[rgb]{0.68,0.00,0.00}{#1}}
\newcommand{\AnnotationTok}[1]{\textcolor[rgb]{0.37,0.37,0.37}{#1}}
\newcommand{\AttributeTok}[1]{\textcolor[rgb]{0.40,0.45,0.13}{#1}}
\newcommand{\BaseNTok}[1]{\textcolor[rgb]{0.68,0.00,0.00}{#1}}
\newcommand{\BuiltInTok}[1]{\textcolor[rgb]{0.00,0.23,0.31}{#1}}
\newcommand{\CharTok}[1]{\textcolor[rgb]{0.13,0.47,0.30}{#1}}
\newcommand{\CommentTok}[1]{\textcolor[rgb]{0.37,0.37,0.37}{#1}}
\newcommand{\CommentVarTok}[1]{\textcolor[rgb]{0.37,0.37,0.37}{\textit{#1}}}
\newcommand{\ConstantTok}[1]{\textcolor[rgb]{0.56,0.35,0.01}{#1}}
\newcommand{\ControlFlowTok}[1]{\textcolor[rgb]{0.00,0.23,0.31}{#1}}
\newcommand{\DataTypeTok}[1]{\textcolor[rgb]{0.68,0.00,0.00}{#1}}
\newcommand{\DecValTok}[1]{\textcolor[rgb]{0.68,0.00,0.00}{#1}}
\newcommand{\DocumentationTok}[1]{\textcolor[rgb]{0.37,0.37,0.37}{\textit{#1}}}
\newcommand{\ErrorTok}[1]{\textcolor[rgb]{0.68,0.00,0.00}{#1}}
\newcommand{\ExtensionTok}[1]{\textcolor[rgb]{0.00,0.23,0.31}{#1}}
\newcommand{\FloatTok}[1]{\textcolor[rgb]{0.68,0.00,0.00}{#1}}
\newcommand{\FunctionTok}[1]{\textcolor[rgb]{0.28,0.35,0.67}{#1}}
\newcommand{\ImportTok}[1]{\textcolor[rgb]{0.00,0.46,0.62}{#1}}
\newcommand{\InformationTok}[1]{\textcolor[rgb]{0.37,0.37,0.37}{#1}}
\newcommand{\KeywordTok}[1]{\textcolor[rgb]{0.00,0.23,0.31}{#1}}
\newcommand{\NormalTok}[1]{\textcolor[rgb]{0.00,0.23,0.31}{#1}}
\newcommand{\OperatorTok}[1]{\textcolor[rgb]{0.37,0.37,0.37}{#1}}
\newcommand{\OtherTok}[1]{\textcolor[rgb]{0.00,0.23,0.31}{#1}}
\newcommand{\PreprocessorTok}[1]{\textcolor[rgb]{0.68,0.00,0.00}{#1}}
\newcommand{\RegionMarkerTok}[1]{\textcolor[rgb]{0.00,0.23,0.31}{#1}}
\newcommand{\SpecialCharTok}[1]{\textcolor[rgb]{0.37,0.37,0.37}{#1}}
\newcommand{\SpecialStringTok}[1]{\textcolor[rgb]{0.13,0.47,0.30}{#1}}
\newcommand{\StringTok}[1]{\textcolor[rgb]{0.13,0.47,0.30}{#1}}
\newcommand{\VariableTok}[1]{\textcolor[rgb]{0.07,0.07,0.07}{#1}}
\newcommand{\VerbatimStringTok}[1]{\textcolor[rgb]{0.13,0.47,0.30}{#1}}
\newcommand{\WarningTok}[1]{\textcolor[rgb]{0.37,0.37,0.37}{\textit{#1}}}

\providecommand{\tightlist}{%
  \setlength{\itemsep}{0pt}\setlength{\parskip}{0pt}}\usepackage{longtable,booktabs,array}
\usepackage{calc} % for calculating minipage widths
% Correct order of tables after \paragraph or \subparagraph
\usepackage{etoolbox}
\makeatletter
\patchcmd\longtable{\par}{\if@noskipsec\mbox{}\fi\par}{}{}
\makeatother
% Allow footnotes in longtable head/foot
\IfFileExists{footnotehyper.sty}{\usepackage{footnotehyper}}{\usepackage{footnote}}
\makesavenoteenv{longtable}
\usepackage{graphicx}
\makeatletter
\def\maxwidth{\ifdim\Gin@nat@width>\linewidth\linewidth\else\Gin@nat@width\fi}
\def\maxheight{\ifdim\Gin@nat@height>\textheight\textheight\else\Gin@nat@height\fi}
\makeatother
% Scale images if necessary, so that they will not overflow the page
% margins by default, and it is still possible to overwrite the defaults
% using explicit options in \includegraphics[width, height, ...]{}
\setkeys{Gin}{width=\maxwidth,height=\maxheight,keepaspectratio}
% Set default figure placement to htbp
\makeatletter
\def\fps@figure{htbp}
\makeatother

\KOMAoption{captions}{tableheading}
\makeatletter
\@ifpackageloaded{tcolorbox}{}{\usepackage[many]{tcolorbox}}
\@ifpackageloaded{fontawesome5}{}{\usepackage{fontawesome5}}
\definecolor{quarto-callout-color}{HTML}{909090}
\definecolor{quarto-callout-note-color}{HTML}{0758E5}
\definecolor{quarto-callout-important-color}{HTML}{CC1914}
\definecolor{quarto-callout-warning-color}{HTML}{EB9113}
\definecolor{quarto-callout-tip-color}{HTML}{00A047}
\definecolor{quarto-callout-caution-color}{HTML}{FC5300}
\definecolor{quarto-callout-color-frame}{HTML}{acacac}
\definecolor{quarto-callout-note-color-frame}{HTML}{4582ec}
\definecolor{quarto-callout-important-color-frame}{HTML}{d9534f}
\definecolor{quarto-callout-warning-color-frame}{HTML}{f0ad4e}
\definecolor{quarto-callout-tip-color-frame}{HTML}{02b875}
\definecolor{quarto-callout-caution-color-frame}{HTML}{fd7e14}
\makeatother
\makeatletter
\makeatother
\makeatletter
\makeatother
\makeatletter
\@ifpackageloaded{caption}{}{\usepackage{caption}}
\AtBeginDocument{%
\ifdefined\contentsname
  \renewcommand*\contentsname{Table of contents}
\else
  \newcommand\contentsname{Table of contents}
\fi
\ifdefined\listfigurename
  \renewcommand*\listfigurename{List of Figures}
\else
  \newcommand\listfigurename{List of Figures}
\fi
\ifdefined\listtablename
  \renewcommand*\listtablename{List of Tables}
\else
  \newcommand\listtablename{List of Tables}
\fi
\ifdefined\figurename
  \renewcommand*\figurename{Figure}
\else
  \newcommand\figurename{Figure}
\fi
\ifdefined\tablename
  \renewcommand*\tablename{Table}
\else
  \newcommand\tablename{Table}
\fi
}
\@ifpackageloaded{float}{}{\usepackage{float}}
\floatstyle{ruled}
\@ifundefined{c@chapter}{\newfloat{codelisting}{h}{lop}}{\newfloat{codelisting}{h}{lop}[chapter]}
\floatname{codelisting}{Listing}
\newcommand*\listoflistings{\listof{codelisting}{List of Listings}}
\makeatother
\makeatletter
\@ifpackageloaded{caption}{}{\usepackage{caption}}
\@ifpackageloaded{subcaption}{}{\usepackage{subcaption}}
\makeatother
\makeatletter
\@ifpackageloaded{tcolorbox}{}{\usepackage[many]{tcolorbox}}
\makeatother
\makeatletter
\@ifundefined{shadecolor}{\definecolor{shadecolor}{rgb}{.97, .97, .97}}
\makeatother
\makeatletter
\makeatother
\ifLuaTeX
  \usepackage{selnolig}  % disable illegal ligatures
\fi
\IfFileExists{bookmark.sty}{\usepackage{bookmark}}{\usepackage{hyperref}}
\IfFileExists{xurl.sty}{\usepackage{xurl}}{} % add URL line breaks if available
\urlstyle{same} % disable monospaced font for URLs
\hypersetup{
  pdftitle={Practica},
  colorlinks=true,
  linkcolor={blue},
  filecolor={Maroon},
  citecolor={Blue},
  urlcolor={Blue},
  pdfcreator={LaTeX via pandoc}}

\title{Practica}
\author{}
\date{}

\begin{document}
\maketitle
\ifdefined\Shaded\renewenvironment{Shaded}{\begin{tcolorbox}[breakable, frame hidden, interior hidden, borderline west={3pt}{0pt}{shadecolor}, enhanced, sharp corners, boxrule=0pt]}{\end{tcolorbox}}\fi

\hypertarget{practica-base-de-datos-co2}{%
\subsection{Practica: base de datos
CO2}\label{practica-base-de-datos-co2}}

\begin{Shaded}
\begin{Highlighting}[]
\FunctionTok{library}\NormalTok{(stats)}
\FunctionTok{library}\NormalTok{(ggplot2)}
\FunctionTok{library}\NormalTok{(tidyverse)}
\end{Highlighting}
\end{Shaded}

\begin{verbatim}
-- Attaching packages --------------------------------------- tidyverse 1.3.2 --
v tibble  3.1.8      v dplyr   1.0.10
v tidyr   1.2.1      v stringr 1.4.1 
v readr   2.1.3      v forcats 0.5.2 
v purrr   0.3.5      
-- Conflicts ------------------------------------------ tidyverse_conflicts() --
x dplyr::filter() masks stats::filter()
x dplyr::lag()    masks stats::lag()
\end{verbatim}

\begin{Shaded}
\begin{Highlighting}[]
\FunctionTok{library}\NormalTok{(latex2exp) }\CommentTok{\# para escribir formulas matematicas con latex}
\FunctionTok{library}\NormalTok{(ggtext)}
\FunctionTok{library}\NormalTok{(extrafont)}
\end{Highlighting}
\end{Shaded}

\begin{verbatim}
Registering fonts with R
\end{verbatim}

\begin{Shaded}
\begin{Highlighting}[]
\CommentTok{\# font\_import()}
\CommentTok{\# loadfonts(device = "win") para importar fuentes}
\end{Highlighting}
\end{Shaded}

\begin{Shaded}
\begin{Highlighting}[]
\NormalTok{co2 }\OtherTok{\textless{}{-}} \FunctionTok{read.csv2}\NormalTok{(}\StringTok{"co2.csv"}\NormalTok{, }\AttributeTok{header =}\NormalTok{ T)}
\FunctionTok{View}\NormalTok{(co2)}
\FunctionTok{str}\NormalTok{(co2)}
\end{Highlighting}
\end{Shaded}

\begin{verbatim}
'data.frame':   28 obs. of  6 variables:
 $ X          : int  1 2 3 4 5 6 7 8 9 10 ...
 $ Type       : chr  "Quebec" "Quebec" "Quebec" "Quebec" ...
 $ Tratamiento: chr  "Control" "Control" "Control" "Control" ...
 $ conc       : int  95 175 250 350 500 675 1000 95 175 250 ...
 $ prom.uptake: chr  "15.27" "30.03" "37.4" "40.37" ...
 $ sd.uptake  : chr  "1.45" "2.57" "2.76" "2.75" ...
\end{verbatim}

\begin{Shaded}
\begin{Highlighting}[]
\NormalTok{quebec}\OtherTok{\textless{}{-}}\NormalTok{ co2 }\SpecialCharTok{\%\textgreater{}\%} \FunctionTok{mutate}\NormalTok{(}\AttributeTok{conc=} \FunctionTok{as.factor}\NormalTok{(conc)) }\SpecialCharTok{\%\textgreater{}\%} 
  \FunctionTok{mutate}\NormalTok{(}\AttributeTok{prom.uptake=} \FunctionTok{as.numeric}\NormalTok{(prom.uptake)) }\SpecialCharTok{\%\textgreater{}\%} \FunctionTok{mutate}\NormalTok{(}\AttributeTok{sd.uptake=} \FunctionTok{as.numeric}\NormalTok{(sd.uptake))}

\FunctionTok{str}\NormalTok{(quebec)}
\end{Highlighting}
\end{Shaded}

\begin{verbatim}
'data.frame':   28 obs. of  6 variables:
 $ X          : int  1 2 3 4 5 6 7 8 9 10 ...
 $ Type       : chr  "Quebec" "Quebec" "Quebec" "Quebec" ...
 $ Tratamiento: chr  "Control" "Control" "Control" "Control" ...
 $ conc       : Factor w/ 7 levels "95","175","250",..: 1 2 3 4 5 6 7 1 2 3 ...
 $ prom.uptake: num  15.3 30 37.4 40.4 39.6 ...
 $ sd.uptake  : num  1.45 2.57 2.76 2.75 3.9 2.35 3.06 3.12 3.15 3.93 ...
\end{verbatim}

\begin{Shaded}
\begin{Highlighting}[]
\FunctionTok{theme\_set}\NormalTok{(}\FunctionTok{theme\_test}\NormalTok{())}

\FunctionTok{ggplot}\NormalTok{(}\AttributeTok{data =}\NormalTok{ quebec, }\FunctionTok{aes}\NormalTok{(}\AttributeTok{x=}\NormalTok{ conc, }\AttributeTok{y=}\NormalTok{prom.uptake ,}\AttributeTok{group =}\NormalTok{ Tratamiento ,}\AttributeTok{colour=}\NormalTok{Tratamiento ))}\SpecialCharTok{+} \FunctionTok{geom\_line}\NormalTok{()}\SpecialCharTok{+}
  \FunctionTok{geom\_point}\NormalTok{(}\AttributeTok{size =} \FloatTok{3.2}\NormalTok{)}\SpecialCharTok{+} \FunctionTok{geom\_errorbar}\NormalTok{(}\FunctionTok{aes}\NormalTok{(}\AttributeTok{ymin=}\NormalTok{prom.uptake}\SpecialCharTok{{-}}\NormalTok{sd.uptake, }\AttributeTok{ymax=}\NormalTok{prom.uptake}\SpecialCharTok{+}\NormalTok{sd.uptake , }\AttributeTok{group =}\NormalTok{Tratamiento ), }\AttributeTok{width=}\NormalTok{.}\DecValTok{1}\NormalTok{)}\SpecialCharTok{+}
  \FunctionTok{scale\_color\_manual}\NormalTok{(}\AttributeTok{values =} \FunctionTok{c}\NormalTok{(}\StringTok{"deepskyblue1"}\NormalTok{,}\StringTok{"firebrick2"}\NormalTok{))}\SpecialCharTok{+} \FunctionTok{facet\_grid}\NormalTok{(Type }\SpecialCharTok{\textasciitilde{}}\NormalTok{ .)}\SpecialCharTok{+}
  \FunctionTok{labs}\NormalTok{(}\AttributeTok{title=} \StringTok{"CO2 en plantas herbáceas"}\NormalTok{, }
\AttributeTok{subtitle=}\FunctionTok{paste0}\NormalTok{(}\StringTok{"Absorción de CO2 de *Echinochloa crus{-}galli* en dos sitios }
\StringTok{(Quebec y Mississippi) }
\StringTok{sometida a dos tratamamientos:"}\NormalTok{,}
\StringTok{"\textless{}span style = \textquotesingle{}color:deepskyblue1\textquotesingle{}\textgreater{}**Control**\textless{}/span\textgreater{}"}\NormalTok{, }\StringTok{" y "}\NormalTok{ , }\StringTok{"\textless{}span style = \textquotesingle{}color:firebrick2\textquotesingle{}\textgreater{}**Enfriado**\textless{}/span\textgreater{}"}\NormalTok{ ) , }\AttributeTok{x=} \FunctionTok{TeX}\NormalTok{(r}\StringTok{"(Concentración CO2 $(ml/L)$)"}\NormalTok{), }\AttributeTok{y=} \FunctionTok{TeX}\NormalTok{(r}\StringTok{"(Absorción CO2 $(\textbackslash{}mu mol/m\^{}\{2\}s)$)"}\NormalTok{), }\AttributeTok{caption =} \StringTok{"Base de datos tomada de la libreria stats (CO2)"}\NormalTok{)}\SpecialCharTok{+}
\FunctionTok{theme}\NormalTok{(}\AttributeTok{strip.background =} \FunctionTok{element\_rect}\NormalTok{(}\AttributeTok{fill =} \StringTok{"white"}\NormalTok{, }\AttributeTok{colour =} \StringTok{"black"}\NormalTok{), }\AttributeTok{strip.text =} \FunctionTok{element\_text}\NormalTok{(}\AttributeTok{face =} \StringTok{"bold"}\NormalTok{, }\AttributeTok{size =} \DecValTok{12}\NormalTok{),}
        \AttributeTok{legend.position =} \FunctionTok{c}\NormalTok{(}\FloatTok{0.1}\NormalTok{, }\FloatTok{0.9}\NormalTok{), }\AttributeTok{plot.subtitle =}\NormalTok{  ggtext}\SpecialCharTok{::}\FunctionTok{element\_markdown}\NormalTok{())}
\end{Highlighting}
\end{Shaded}

\begin{figure}[H]

{\centering \includegraphics{CO2_files/figure-pdf/unnamed-chunk-3-1.pdf}

}

\end{figure}

\begin{Shaded}
\begin{Highlighting}[]
\FunctionTok{ggplot}\NormalTok{(}\AttributeTok{data =}\NormalTok{ quebec, }\FunctionTok{aes}\NormalTok{(}\AttributeTok{x=}\NormalTok{ conc, }\AttributeTok{y=}\NormalTok{prom.uptake ,}\AttributeTok{group =}\NormalTok{ Tratamiento ,}\AttributeTok{colour=}\NormalTok{Tratamiento ))}\SpecialCharTok{+} \FunctionTok{geom\_line}\NormalTok{()}\SpecialCharTok{+}
  \FunctionTok{geom\_point}\NormalTok{(}\AttributeTok{size =} \FloatTok{3.2}\NormalTok{)}\SpecialCharTok{+} \FunctionTok{geom\_errorbar}\NormalTok{(}\FunctionTok{aes}\NormalTok{(}\AttributeTok{ymin=}\NormalTok{prom.uptake}\SpecialCharTok{{-}}\NormalTok{sd.uptake, }\AttributeTok{ymax=}\NormalTok{prom.uptake}\SpecialCharTok{+}\NormalTok{sd.uptake , }\AttributeTok{group =}\NormalTok{Tratamiento ), }\AttributeTok{width=}\NormalTok{.}\DecValTok{1}\NormalTok{)}\SpecialCharTok{+}
  \FunctionTok{scale\_color\_manual}\NormalTok{(}\AttributeTok{values =} \FunctionTok{c}\NormalTok{(}\StringTok{"deepskyblue1"}\NormalTok{,}\StringTok{"firebrick2"}\NormalTok{))}\SpecialCharTok{+} \FunctionTok{facet\_grid}\NormalTok{(Type }\SpecialCharTok{\textasciitilde{}}\NormalTok{ .)}\SpecialCharTok{+}
  \FunctionTok{labs}\NormalTok{(}\AttributeTok{x=} \FunctionTok{TeX}\NormalTok{(r}\StringTok{"(Concentración CO2 $(ml/L)$)"}\NormalTok{), }\AttributeTok{y=} \FunctionTok{TeX}\NormalTok{(r}\StringTok{"(Absorción CO2 $(\textbackslash{}mu mol/m\^{}\{2\}s)$)"}\NormalTok{), }\AttributeTok{caption =} \StringTok{"Base de datos tomada de la libreria stats (CO2)"}\NormalTok{)}\SpecialCharTok{+}
  \FunctionTok{theme}\NormalTok{(}\AttributeTok{strip.background =} \FunctionTok{element\_rect}\NormalTok{(}\AttributeTok{fill =} \StringTok{"white"}\NormalTok{, }\AttributeTok{colour =} \StringTok{"black"}\NormalTok{), }\AttributeTok{strip.text =} \FunctionTok{element\_text}\NormalTok{(}\AttributeTok{face =} \StringTok{"bold"}\NormalTok{, }\AttributeTok{size =} \DecValTok{12}\NormalTok{),}
        \AttributeTok{legend.position =} \FunctionTok{c}\NormalTok{(}\FloatTok{0.1}\NormalTok{, }\FloatTok{0.9}\NormalTok{), }\AttributeTok{text =} \FunctionTok{element\_text}\NormalTok{(}\AttributeTok{size =} \DecValTok{14}\NormalTok{), }
        \AttributeTok{axis.text =} \FunctionTok{element\_text}\NormalTok{(}\AttributeTok{size =} \DecValTok{14}\NormalTok{), }
        \AttributeTok{legend.text =} \FunctionTok{element\_text}\NormalTok{(}\AttributeTok{size =} \DecValTok{12}\NormalTok{) )}
\end{Highlighting}
\end{Shaded}

\begin{figure}[H]

{\centering \includegraphics{CO2_files/figure-pdf/unnamed-chunk-3-2.pdf}

}

\end{figure}

\begin{tcolorbox}[enhanced jigsaw, rightrule=.15mm, coltitle=black, opacitybacktitle=0.6, arc=.35mm, toptitle=1mm, breakable, bottomrule=.15mm, leftrule=.75mm, colback=white, title=\textcolor{quarto-callout-note-color}{\faInfo}\hspace{0.5em}{La library(latex2exp)}, bottomtitle=1mm, titlerule=0mm, toprule=.15mm, colframe=quarto-callout-note-color-frame, colbacktitle=quarto-callout-note-color!10!white, left=2mm, opacityback=0]
Te permite escribir formulas con lenguaje latex. Para eso solo utilizas
x= TeX(r''(texto \$ formula matemática \$)``)
\end{tcolorbox}

\hypertarget{raincloud-plot-con-ggplot2}{%
\subsection{Raincloud plot con
ggplot2}\label{raincloud-plot-con-ggplot2}}

Los diagramas de nubes de lluvia se presentaron en 2019 como un enfoque
para superar los problemas de ocultar la verdadera distribución de datos
al trazar barras con barras de error (también conocidas como
\textbf{\emph{diagramas de dinamita}} ) o diagramas de caja. En cambio,
los diagramas de nubes de lluvia combinan varios tipos de gráficos para
visualizar los datos sin procesar, la distribución de los datos como
densidad y las estadísticas de resumen clave al mismo tiempo.

Para este ejercicio vamos a utilizar las siguientes librerías :

\begin{Shaded}
\begin{Highlighting}[]
\FunctionTok{library}\NormalTok{(ggplot2)}
\FunctionTok{library}\NormalTok{(ggforce)}
\FunctionTok{library}\NormalTok{(ggdist)}
\FunctionTok{library}\NormalTok{(gghalves)}
\end{Highlighting}
\end{Shaded}

\begin{Shaded}
\begin{Highlighting}[]
\FunctionTok{library}\NormalTok{(stats)}
\FunctionTok{library}\NormalTok{(tidyverse)}
\FunctionTok{data}\NormalTok{(}\StringTok{"iris"}\NormalTok{)}
\FunctionTok{str}\NormalTok{(iris)}
\end{Highlighting}
\end{Shaded}

\begin{verbatim}
'data.frame':   150 obs. of  5 variables:
 $ Sepal.Length: num  5.1 4.9 4.7 4.6 5 5.4 4.6 5 4.4 4.9 ...
 $ Sepal.Width : num  3.5 3 3.2 3.1 3.6 3.9 3.4 3.4 2.9 3.1 ...
 $ Petal.Length: num  1.4 1.4 1.3 1.5 1.4 1.7 1.4 1.5 1.4 1.5 ...
 $ Petal.Width : num  0.2 0.2 0.2 0.2 0.2 0.4 0.3 0.2 0.2 0.1 ...
 $ Species     : Factor w/ 3 levels "setosa","versicolor",..: 1 1 1 1 1 1 1 1 1 1 ...
\end{verbatim}

\begin{Shaded}
\begin{Highlighting}[]
\FunctionTok{theme\_set}\NormalTok{(}\FunctionTok{theme\_bw}\NormalTok{())}
\FunctionTok{ggplot}\NormalTok{(iris, }\FunctionTok{aes}\NormalTok{(Species, Sepal.Width,}\AttributeTok{fill=}\NormalTok{ Species, }\AttributeTok{color=}\NormalTok{ Species ))}\SpecialCharTok{+} 
\NormalTok{  ggdist}\SpecialCharTok{::}\FunctionTok{stat\_halfeye}\NormalTok{(}\FunctionTok{aes}\NormalTok{(}\AttributeTok{fill =}\NormalTok{ Species), }\AttributeTok{adjust =}\NormalTok{ .}\DecValTok{5}\NormalTok{, }\AttributeTok{width =}\NormalTok{ .}\DecValTok{3}\NormalTok{,}
                       \AttributeTok{.width =} \DecValTok{0}\NormalTok{, }\AttributeTok{justification =} \SpecialCharTok{{-}}\NormalTok{.}\DecValTok{3}\NormalTok{, }\AttributeTok{point\_colour =} \ConstantTok{NA}\NormalTok{)}\SpecialCharTok{+} 
  \FunctionTok{geom\_boxplot}\NormalTok{(}\AttributeTok{width =}\NormalTok{ .}\DecValTok{1}\NormalTok{, }\AttributeTok{outlier.shape =} \ConstantTok{NA}\NormalTok{, }\AttributeTok{alpha=}\NormalTok{ .}\DecValTok{5}\NormalTok{) }\SpecialCharTok{+}\FunctionTok{geom\_point}\NormalTok{(}\FunctionTok{aes}\NormalTok{(}\AttributeTok{x =} \FunctionTok{as.numeric}\NormalTok{(Species)}\SpecialCharTok{{-}}\NormalTok{.}\DecValTok{15}\NormalTok{, }\AttributeTok{y =}\NormalTok{ Sepal.Width, }\AttributeTok{colour =}\NormalTok{ Species),}\AttributeTok{position =} \FunctionTok{position\_jitter}\NormalTok{(}\AttributeTok{width =}\NormalTok{ .}\DecValTok{05}\NormalTok{), }\AttributeTok{size =} \DecValTok{2}\NormalTok{, }\AttributeTok{shape =} \DecValTok{20}\NormalTok{)}
\end{Highlighting}
\end{Shaded}

\begin{figure}[H]

{\centering \includegraphics{CO2_files/figure-pdf/unnamed-chunk-6-1.pdf}

}

\end{figure}



\end{document}
